\section{Exercice 2}
\subsection{Question 1}
In order to show that the tilling problem is in NP, we can show that there exists a polynomial algorithm that check if the given assignement $\Theta$ is a tillement.

Input : A set of Tile $T$ , a grid size $N>0$ encoded in unary, and $\Theta = N \times N \rightarrow T$ an assignment.

Question : Is $\Theta$ a valid tilling by $T$ for the grid $N \times N$

\begin{lstlisting}
checking_till(T,N,Theta) {
	for(i = 0; i<N-1 ; i++){
		for(j = 0; j<N-1; j ++){
			if(Theta(i,j) not in T) return false;
			else if (Theta(i,j)[left] <> Theta(i,j)[right]) return false;
			else if (Theta(i,j)[bot] <> Theta(i,j+1)[top]) return false;
		}
	}
	return true;
}
\end{lstlisting}

checking\_till does at most $3*(N-1)^2$ operations then it is polynomial.
\\And obtaining a tile in Theta can be done easily in linear time among $n$
\\Hence the tilling probleme is in NP.

\subsection{Question 2}

Input: A one-tape nondeterministic Turing machine $M$ and a time bound $B > 0$ encoded in unary.
\\Question: Does $M$ have a computation of length $B$ on the empty input $\epsilon$ ?

Input: A tile set $T$ and a grid size $N > 0$ encoded in unary.
\\Question: Does there exist a tiling of the $N \times N$ square grid by $T$ ?


We tweak the machine $M$ such that we create a machine $M'$ which emulate $B-1$ steps of $M$ and then go to a special $q_{loose}$ state if the next state of the $M$ computation wasn't $accept$ or $reject$, on $q_{loose}$, $M'$ loops indefinetly.
This construction is easily buildable in polynamial time in the way as the one in the previous question (As we have to count steps). 
Such machine verify that $M'$ admit a computation of lenght $B$ on $\epsilon$ if and only if $M$ admit a computaton of lenght $B$ on $\epsilon$
Moreover every computation that doesn't end in $B$ steps, land on $q_loose$ after $B$ steps.
We will now assmume that the Machine considered is like $M'$.

Let $M = <Q,\delta,q_0,\sum,\{b,\$\}>$

We build the set of tile $T$. 
\\First $N= B+1$, each row represent a state of the machine. ie. $B$ charactere from the alphabet and a state. As the computation of $M$ is bouded by $B$, their will be at most $B$  different character from $b$.


For each $a \in \sum \cup \{b,\$\}$ we add to $T$
\\$<f,a,f,a>$

The intuition behind $f$ is the creation of a special color to force any element of the state to remain the same between two row (exept those concern by the transtion).

For each $q \in Q, a \in \sum\cup \{b,\$\}$ if $(p,c,\downarrow) \in \delta (q,a)$ and $p\neq q_{loose}$ then we add to $T$
\\$<q_p,p,f ,q>$ and (head tile)
\\$<f,c,q_p,a>$ (right tile)

The intuition here is to create a Tile which force her left side to remain unchange thanks to $f$, make the state move from $q$ to $p$ and warn the right tile representing $a$ that a transition from $q$ to $p$ as occured, then we add an another tile which reprensent the fact that $a$ can become $c$ when a  $q$ to $p$ transition occured.
If $q$ can not become $p$ when reading $a$ then the tilling will stop there as there won't be any available tile to match $q_p$ and $a$.

For each $q \in Q, a \in \sum\cup \{b,\$\}$ if $(p,c,\rightarrow) \in \delta (q,a)$ and $p\neq q_{loose}$ then we add to $T$
\\$<q_c^{\rightarrow},c, f,q>$ and  (head tile)
\\$<f,p,q_c^{\rightarrow},a>$ (right tile)

The intuition here is to create a Tile which will make the head move on and write a $c$ at her place. This also warn her right tile that it moves from $q$ to $c$.
It also add a tile representing the new head on the state $p$.
If $q$ can not become $c$ when reading $a$ then the tilling will stop as $<f,p,q_c^{\rightarrow},a>$ wont exist.


For each $q \in Q, a \in \sum\cup \{b,\$\}$ if $(p,c,\leftarrow) \in \delta (q,a)$ and $p\neq q_{loose}$ then we add to $T$
\\For each $d \in \sum\cup \{b,\$\}$
\\$<q_c^a,d,q_c^a\times d,q>$ and (head tiles)
\\$<q_c^a\times d,p,f,d>$ and (left tiles)
\\$<f,c,q_c^a,a>$ (right tile)

Here it's a bit tricky has we have to imagine both what the left and right tile are.
In order to be able to do so we have to create many colors.
If there isn't a transition $(p,c,\leftarrow) \in \delta (q,a)$ then $<q_c^a\times d,p,f,d>$ won't exists. If $a$ wasn't the right tile then we can't apply $<f,c,q_c^a,a>$, if $d$ wasn't the left tile then we can't apply $<q_c^a\times d,p,f,d>$.

We already create one tile for each element of $\sum\cup \{b,\$\}$, 2 tiles for each $\downarrow$ transition, 2 tiles for each $\rightarrow$ transition, and $2 * |\sum\cup \{b,\$\}| + 1$ tiles for each $\leftarrow$ transition.
Moreover we also have at most create : $|\sum\cup \{b,\$\}| + |Q|*|Q| + |\sum\cup \{b,\$\}| * |Q| + |\sum\cup \{b,\$\}|^2 * |Q| + |\sum\cup \{b,\$\}|^3 * |Q|$ different colors.
\\Both of those values are polynomial.

We now have have to constrain the first row thanks to $t_0$ to be $\epsilon q_0 \$ b ....b$ in order so we introduce a special color $i$ only use on the first row and special tiles: $t_0 = <\$,q_0,i,i>$ , $<i,\$,\$,i>$ ,and $<i,b,i,i>$.
Here we have a bit of loose in generality because i can't find a way to deal with a set of possible initial states. however we can always add multiple $\epsilon$ transition (ie. which doesn't change the tape) from  $q_0$ to the other member of the initial set. 


As a formal proof will take too long, i will try to convince you.


If $M$ have a compution of length $B$ on  $\epsilon$.
Consider such computation $r$, as we can add as many column on the first row of the tilling we add $B-1$ tile of $b$, then we can respect the computation $r$ to choose the tilling. It is always possible as we build the set of tile $T$ in order to. We can do so $B$ times resulting in $B+1 \times B+1$ tilling. It 's important to notice that we won't need more $b$ than $B$ cause as the computation is bound by $B$ there is at most $B$ $\rightarrow$ transition.

Considere there is no computation of lenght $B$ on $\epsilon$.
Suppose there exists a $B+1 \times B+1$ tilling with the $T$,
Then inductivly we can prove that each row of the tilling can represent a state of the machine. (Indeed it's clear for the first one as $t_0$ and $i$ insure it, and as to put a till above the head we have to respect condition on $\delta$ it also true for the row above)
Then if we are able to tile $B$ row that mean that is able to perform $B-1$ steps.
Moreover, as we didn't add any tile for $q_{loose}$ if we are able to tile the last row then, the last row contain a tile representing $accept$ or $reject$.
(If we didn't tweak the machine $M$ beforehand any computation longer then $B$ which induce a valid tilling)
Hence the run represent by the row is a valid run for $M$ of lenght $B$ which computs $\epsilon$ (as the first row has to be of form $q_0\$bbb...bb$).

The building of $T$ can be performed in polynomial has, there is a polynomial number of tiles in $T$, and $N=B+1$ is compute linearly.

As a conclusion the tilling problem is NP-Hard, hence it is NP-Complete.